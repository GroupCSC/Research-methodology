\documentclass[12pt]{article}
\begin{document}


\section*{\centerline{POACHING IN UGANDA}}

\subsection*{Problem statement}

This has been a continous problem especially in queen elizabeth
 where most of the poaching happens due to animals crossing to DRC
where armed groups hunt them for money to support their activities


There are also cases of the locals competing with space and some times
 encroachment by both people and animals which lead to more animals getting killed

According to the investigatornews.com, in 2015  six elephants were killed
in queen elizabeth alone which led to registration of the highest number of
elephant deaths that year

The other national parks like murchsion falls have also seen an amount of poaching which was 
mainly carried out by the LRA rebels but dropped when the were driven out


\subsection*{Objectives}
\begin{itemize}
\item To reduce the number of animals being poached reducing the chances of extinction

\item Protect the major source of revenue to the country(tourism)

\item To make it easier to catch the poachers and their accomplices \ldots
\end{itemize}

\subsection*{Solutions}
\begin{itemize}
\item To add trackers to elephant herds to monitor their movement

\item Use of drones to follow some species when they cross into dangerous territory\ldots
\end{itemize}

We used\textbf{ applied research} to identify the problem, \textbf{analytical research} to get the 
numbers of elephant deaths each year in
uganda which were roughly 3 elephants a year according 
to travelhemispheres.com and used \textbf{quantitative research} to and 
established that there are over 5000 
elephants in uganda
\end{document}